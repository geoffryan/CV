% LaTeX Curriculum Vitae Template
%
% Copyright (C) 2004-2009 Jason Blevins <jrblevin@sdf.lonestar.org>
% http://jblevins.org/projects/cv-template/
%
% You may use use this document as a template to create your own CV
% and you may redistribute the source code freely. No attribution is
% required in any resulting documents. I do ask that you please leave
% this notice and the above URL in the source code if you choose to
% redistribute this file.

\documentclass[letterpaper]{article}

\usepackage{hyperref}
\usepackage{geometry}

% Comment the following lines to use the default Computer Modern font
% instead of the Palatino font provided by the mathpazo package.
% Remove the 'osf' bit if you don't like the old style figures.
\usepackage[T1]{fontenc}
\usepackage[sc,osf]{mathpazo}

% Set your name here
\def\name{Geoffrey Ryan}

% Replace this with a link to your CV if you like, or set it empty
% (as in \def\footerlink{}) to remove the link in the footer:
\def\footerlink{}

% The following metadata will show up in the PDF properties
\hypersetup{
  colorlinks = true,
  urlcolor = black,
  pdfauthor = {\name},
  pdfkeywords = {physics, astronomy, astrophysics},
  pdftitle = {\name: Curriculum Vitae},
  pdfsubject = {Curriculum Vitae},
  pdfpagemode = UseNone
}

\geometry{
  body={6.5in, 8.5in},
  left=1.0in,
  top=1.25in
}

% Customize page headers
\pagestyle{myheadings}
\markright{\name}
\thispagestyle{empty}

% Custom section fonts
\usepackage{sectsty}
\sectionfont{\rmfamily\mdseries\scshape\Large}
\subsectionfont{\rmfamily\mdseries\large}

% Other possible font commands include:
% \ttfamily for teletype,
% \sffamily for sans serif,
% \bfseries for bold,
% \scshape for small caps,
% \normalsize, \large, \Large, \LARGE sizes.

% Don't indent paragraphs.
\setlength\parindent{0em}

% Make lists without bullets
\renewenvironment{itemize}{
  \begin{list}{}{
    \setlength{\leftmargin}{1.5em}
  }
}{
  \end{list}
}

\begin{document}

% Place name at left
{\huge \name}

% Alternatively, print name centered and bold:
%\centerline{\huge \bf \name}

\vspace{0.25in}

\begin{minipage}{0.45\linewidth}
  New York University \\
  Physics Department\\
  4 Washington Place\\
  New York, NY 10003-6603
\end{minipage}
\begin{minipage}{0.45\linewidth}
  \begin{tabular}{ll}
    Phone: & 646-531-5867 \\
    Email: &  gsr257@nyu.edu \\
    Homepage: & \href{http://geoffryan.space/}{\tt http://geoffryan.space/} \\
    GitHub: & \href{http://github.com/geoffryan/}{\tt http://github.com/geoffryan/} \\
  \end{tabular}
\end{minipage}

%Education
\section*{Education}
\begin{itemize}
\item \begin{tabular}{ll}
2017 & Ph.D. Physics, New York University, Advisor: Andrew MacFadyen \\
\end{tabular}

\item \begin{tabular}{ll}
2011 &  M.Sc. Physics, University of Alberta, Advisor: Alexander Penin \\
\end{tabular}

\item \begin{tabular}{ll}
2009 &  B.Sc. (Hons) Mathematical Physics, University of Alberta \\
\end{tabular}
\end{itemize}

%\section*{Professional Appointments} SOME DAY....

%Publications
\section*{Publications}
\subsection*{Journal Articles}
\begin{itemize}
\item \begin{tabular}{ll}
2016 & {\bf Ryan, G.}, and MacFadyen, A. "Minidisks in Binary Black Hole Accretion" \\ 
	& Submitted to \emph{ApJ} and arXiv Oct. 31
	%& "Gamma Ray Bursts Are Observed Off-Axis."  \emph{ApJ} 799 (2015) 3, [arXiv:astro-ph/1405.5516] \\
\end{tabular}

\item \begin{tabular}{ll}
2015 & {\bf Ryan, G.}, van Eerten, H., MacFadyen, A., and Zhang, B.B. \\ 
	& "Gamma Ray Bursts Are Observed Off-Axis."  \emph{ApJ} 799 (2015) 3, [arXiv:astro-ph/1405.5516] \\
\end{tabular}

\item \begin{tabular}{ll}
2015 & Zhang, B.B., van Eerten, H., Burrows, D., {\bf Ryan, G.}, {Evans}, P.A., Racusin, J., Troja, E., and MacFadyen, A. \\
 & "Revisiting The GRB Jet-Break Problem With CHANDRA Deep Follow-up Data."  \\
&  \emph{ApJ} 806 (2015) 15, [arXiv:astro-ph/1405.4867] \\
\end{tabular}

\item \begin{tabular}{ll}
2011 & Penin, A., and {\bf Ryan, G.}  "Two-Loop Electroweak Corrections To High Energy Large-Angle \\& Bhabha Scattering." \emph{JHEP} 1111 (2011) 081, [arXiv:hep-ph/1112.2171] \\
\end{tabular}
\end{itemize}

\subsection*{Conference Proceedings}
\begin{itemize}
\item \begin{tabular}{ll}
2013 & {\bf Ryan, G.}, van Eerten, H., and MacFadyen, A. "Fitting Afterglows With Multi-Dimensional Simulations." \\& in Proceedings of the 7th Huntsville Gamma-Ray Burst Symposium, Nashville, Tennessee, USA, 2013, \\& edited by N. Gehrels, M. S. Briggs and V. Connaughton, eConf C1304143, 30, 2013 \\
\end{tabular}
\end{itemize}

%Invited Talks  YET!
\section*{Invited Talks}
\begin{itemize}
\item \begin{tabular}{ll}
2014 & "Gamma-Ray Bursts Are Observed Off-Axis." Institute for Theory and Computation Seminar  \\
	& Harvard-Smithsonian Center for Astrophysics, Cambridge, Massachusetts, USA.  September 23
\end{tabular}
\end{itemize}

%Conferences!
\section*{Conference and Workshop Participation}
\begin{itemize}
\item \begin{tabular}{ll}
2016 & Poster "Minidiscs in Circumbinary Black Hole Accretion" \\
	& 21st International Conference on General Relativity and Gravitation \\ 
	& Columbia University, New York City, USA. July 10-July 15
\end{tabular}

\item \begin{tabular}{ll}
2014 & International Summer School on Astro-Computing - Nuclear and Neutrino Astrophsyics \\ 
	& San Diego Supercomputing Center, University of California, San Diego, USA. July 21-August 1
\end{tabular}

\item \begin{tabular}{ll}
2013 & Poster "Fitting Afterglows With Multi-Dimensional Simulations." \\ 
	&7th Huntsville Gamma-Ray Burst Symposium, Nashville, Tennessee, USA.  April 14-18 \\
\end{tabular}

\item \begin{tabular}{ll}
2010 & SLAC Summer Institute: Neutrinos - Nature's Mysterious Messengers, \\ 
	 & SLAC National Accelerator Laboratory, Menlo Park, California, USA.  August 2-13\\
\end{tabular}
\end{itemize}

%Departmental Talks
%\section*{Departmental Talks}

%Research Experience
%\section*{Research Experience}


%Awards
\section*{Fellowships and Awards}
\begin{itemize}
\item \begin{tabular}{ll}
2015 & Dean's Dissertation Fellowship \\
& New York University, Fall 2015 - Summer 2016 \\
\end{tabular}

\item \begin{tabular}{ll}
2015 & Dean's Outstanding Graduate Student Teaching Award, New York University
\end{tabular}

\item \begin{tabular}{ll}
2014 & James Arthur Fellowship \\
& New York University, Fall 2014 - Summer 2015 \\
\end{tabular}

\item \begin{tabular}{ll}
2013 & James Arthur Fellowship \\
& New York University, Fall 2013 - Summer 2014 \\
\end{tabular}

\item \begin{tabular}{ll}
2011 & Henry M. MacCracken Fellowship \\
& New York University, Fall 2011 - Fall 2015 \\
\end{tabular}

\item \begin{tabular}{ll}
2011 & Graduate Student Teaching Award, University of Alberta \\
\end{tabular}

\item \begin{tabular}{ll}
2011 & Graduate Student Scholarship, Government of Alberta \\
\end{tabular}

%\item \begin{tabular}{ll}
%2008 & NSERC Undergraduate Summer Research Award
%\end{tabular}
%
%\item \begin{tabular}{ll}
%2008 & Jason Lang Award, University of Alberta
%\end{tabular}
%
%\item \begin{tabular}{ll}
%2007 & NSERC Undergraduate Summer Research Award
%\end{tabular}
%
%\item \begin{tabular}{ll}
%2007 & Academic Award, University of Alberta
%\end{tabular}
%
%\item \begin{tabular}{ll}
%2007 & Jason Lang Award, University of Alberta
%\end{tabular}
%
%\item \begin{tabular}{ll}
%2006 & NSERC Undergraduate Summer Research Award
%\end{tabular}
%
%\item \begin{tabular}{ll}
%2006 & Jason Lang Award, University of Alberta
%\end{tabular}
%
%\item \begin{tabular}{ll}
%2005 & Academic Excellence, University of Alberta
%\end{tabular}
%
%\item \begin{tabular}{ll}
%2005 & Entrance Scholarship, University of Alberta
%\end{tabular}

\end{itemize}

%Teaching Experience
\section*{Teaching Experience}
\subsection*{New York University}
\begin{itemize}
\item \begin{tabular}{ll}
2016 & Graduate Computational Physics \\
2016 & General Relativity \\
2014 & Graduate Computational Physics \\
2013 & Graduate Quantum Mechanics II \\
2012 & Graduate Quantum Mechanics I \\
2012 & Introductory Experimental Physics II \\
\end{tabular}
\end{itemize}
\subsection*{University of Alberta}
\begin{itemize}
\item \begin{tabular}{ll}
2011 & Physics 146 \\
2010 & Physics 144 \\
2010 & Physics 126 \\
2009 & Physics 124 \\
\end{tabular}
\end{itemize}

%Computers
%\section*{Computer Languages}
%\begin{itemize}
%\item C
%\item Python
%\item Java
%\item Mathematica
%\end{itemize}



\bigskip

% Footer
%\begin{center}
  %\begin{footnotesize}
   % Last updated: \today \\
    %\href{\footerlink}{\texttt{\footerlink}}
  %\end{footnotesize}
%\end{center}

\end{document}
