% LaTeX Curriculum Vitae Template
%
% Copyright (C) 2004-2009 Jason Blevins <jrblevin@sdf.lonestar.org>
% http://jblevins.org/projects/cv-template/
%
% You may use use this document as a template to create your own CV
% and you may redistribute the source code freely. No attribution is
% required in any resulting documents. I do ask that you please leave
% this notice and the above URL in the source code if you choose to
% redistribute this file.

\documentclass[letterpaper]{article}

\usepackage{hyperref}
\usepackage{geometry}

% Comment the following lines to use the default Computer Modern font
% instead of the Palatino font provided by the mathpazo package.
% Remove the 'osf' bit if you don't like the old style figures.
\usepackage[T1]{fontenc}
\usepackage[sc,osf]{mathpazo}

% Set your name here
\def\name{Geoffrey Ryan}

% Replace this with a link to your CV if you like, or set it empty
% (as in \def\footerlink{}) to remove the link in the footer:
\def\footerlink{}

% The following metadata will show up in the PDF properties
\hypersetup{
  colorlinks = true,
  urlcolor = black,
  pdfauthor = {\name},
  pdfkeywords = {physics, astronomy, astrophysics},
  pdftitle = {\name: Curriculum Vitae},
  pdfsubject = {Curriculum Vitae},
  pdfpagemode = UseNone
}

\geometry{
  body={6.5in, 8.5in},
  left=1.0in,
  top=1.25in
}

% Customize page headers
\pagestyle{myheadings}
\markright{\name}
\thispagestyle{empty}

% Custom section fonts
\usepackage{sectsty}
\sectionfont{\rmfamily\mdseries\scshape\Large}
\subsectionfont{\rmfamily\mdseries\large}

% Other possible font commands include:
% \ttfamily for teletype,
% \sffamily for sans serif,
% \bfseries for bold,
% \scshape for small caps,
% \normalsize, \large, \Large, \LARGE sizes.

% Don't indent paragraphs.
\setlength\parindent{0em}

% Make lists without bullets
\renewenvironment{itemize}{
  \begin{list}{}{
    \setlength{\leftmargin}{1.5em}
  }
}{
  \end{list}
}

\begin{document}

% Place name at left
{\huge \name}

% Alternatively, print name centered and bold:
%\centerline{\huge \bf \name}

\vspace{0.25in}

\begin{minipage}{0.45\linewidth}
  University of Maryland \\
  Department of Astronomy\\
  1113 PSC Bldg. 415\\
  College Park, MD 20742
\end{minipage}
\begin{minipage}{0.45\linewidth}
  \begin{tabular}{ll}
    Phone: & 646-531-5867 \\
    Email: &  gsryan@umd.edu \\
    Homepage: & \href{http://geoffryan.space/}{\tt http://geoffryan.space/} \\
    GitHub: & \href{http://github.com/geoffryan/}{\tt http://github.com/geoffryan/} \\
  \end{tabular}
\end{minipage}

%Education
\section*{Education}
\begin{itemize}
\item \begin{tabular}{ll}
2017 & Ph.D. Physics, New York University, Advisor: Andrew MacFadyen \\
	& Thesis: Numerical Simulations Of Black Hole Accretion
\end{tabular}

\item \begin{tabular}{ll}
2011 &  M.Sc. Physics, University of Alberta, Advisor: Alexander Penin \\
	& Thesis: The High Energy Logarithms in Two Loop Electroweak Bhabha Scattering
\end{tabular}

\item \begin{tabular}{ll}
2009 &  B.Sc. (Hons) Mathematical Physics, University of Alberta \\
\end{tabular}
\end{itemize}

%Jobs
\section*{Professional Appointments} % SOME DAY.... DAY!
\begin{itemize}
\item \begin{tabular}{ll}
2017-2020 & Joint Space-Science Institute Prize Fellow \\
	 & University of Maryland College Park \& NASA Goddard Space Flight Center\\
\end{tabular}
\item \begin{tabular}{ll}
2020- & Post Doctoral Associate, Astroparticle Physics Lab\\
	  & University of Maryland College Park \& NASA Goddard Space Flight Center\\
\end{tabular}
\end{itemize}

%Publications
\section*{Publications}
\subsection*{Journal Articles}
\begin{itemize}

\item \begin{tabular}{ll}
2019 & {\bf {Ryan}, G.}, {van Eerten}, H., {Piro}, L., and {Troja}, E. \\
	& ``Gamma-Ray Burst Afterglows In The Multi-Messenger Era: Numerical Models and Closure Relations'' \\
	& \emph{ApJ} submitted, [\href{https://arxiv.org/abs/1909.11691}{arXiv:astro-ph/1909.11691}]
\end{tabular}

\item \begin{tabular}{ll}
2019 & {Troja}, E., {Castro-Tirado}, A.J.,  {Becerra Gonzalez}, J.,
         {Hu}, Y., {\bf {Ryan}, G.S.}, {Cenko}, S.B., {Ricci}, R., \\ &
         {Novara}, G., {Sanchez-Ramirez}, R., {Acosta-Pulido}, J.A.,
         {Caballero Garcia}, M.D., {Guziy}, S.,  \\ &
         {Jeong}, S.,
         {Lien}, A.Y., {Marquez}, I., {Pandey}, S.B., {Park}, I.H.,
         {Tello}, J.C., {Sakamoto}, T., {Sokolov}, I.V., \\ &
         {Sokolov}, V.V., {Tiengo}, A., {Valeev}, A.F.,
         {Zhang}, B.B., and {Veilleux}, S. \\ &``The Afterglow And Kilonova Of The Short GRB 160821B'' \\ &
         \emph{MNRAS} 489 (2019) 2104-2116, [\href{https://arxiv.org/abs/1905.01290}{arXiv:astro-ph/1905.01290}]
\end{tabular}

\item \begin{tabular}{ll}
2019 & Troja, E., van Eerten, H., {\bf Ryan, G.}, Ricci, R., Burgess, J.M., Wieringa, M., Piro, L., Cenko, S.B.,\\
	& and Sakamoto, T. ``A Year In The Life Of GW170817: The Rise And Fall Of A Structured Jet From\\
	&  A Binary Neutron Star Merger'' \emph{MNRAS} 489 (2019) 1919-1926, [\href{https://arxiv.org/abs/1808.06617}{arXiv:astro-ph/1808.06617}]
\end{tabular}

\item \begin{tabular}{ll}
2019 &  Piro, L., Troja, E., Zhang, B., {\bf Ryan, G.}, van Eerten, H. , Ricci, R., Wieringa, M. H., Tiengo, A.,\\
	& Butler, N.R., Cenko, S.B., Fox, O.D., Kandrika, H.G., Novara, G., Rossi, A., and Sakamoto, T. \\
	& ``A Long-Lived Neutron Star Merger Remnant In GW170817: Constraints And Clues From\\
	& X-Ray Observations'' \emph{MNRAS} 483 (2019) 1912-1921, [\href{https://arxiv.org/abs/1810.04664}{arXiv:astro-ph/1810.04664}]
\end{tabular}

\item \begin{tabular}{ll}
2018 & Troja, E., {\bf Ryan, G.}, Piro, L., van Eerten, H., Cenko, S.B., Yoon, Y., Lee, S.K., Im, M., Sakamoto, T., \\
	& Gatkine, P., Kutyrev, A., and Veilleux, S. \\
	& ``A Luminous Blue Kilonova And An Off-Axis Jet From A Compact Binary Merger At z = 0.1341''\\
	& \emph{\bf Nature Communications} 9 (2018) id. 4089, [\href{https://arxiv.org/abs/1806.10624}{arXiv:astro-ph/1806.10624}]
\end{tabular}

\item \begin{tabular}{ll}
2018 & Troja, E., Piro, L., {\bf Ryan, G.}, van Eerten, H., Ricci, R., Wieringa, M.H., Lotti, S., Sakamoto, T.,\\
	& and Cenko, S.B. ``The Outflow Structure Of GW170817 From Late-Time Broad-Band Observations'' \\
	& \emph{MNRAS} 478 (2018) L18-L23, [\href{https://arxiv.org/abs/1801.06516}{arXiv:astro-ph/1801.06516}]
\end{tabular}

\item \begin{tabular}{ll}
2017 & {\bf Ryan, G.}, and MacFadyen, A. ``Minidisks in Binary Black Hole Accretion'' \\ 
	& \emph{ApJ} 835 (2017) 199, [\href{https://arxiv.org/abs/1611.00341}{arXiv:astro-ph/1611.00341}]
\end{tabular}

\item \begin{tabular}{ll}
2015 & {\bf Ryan, G.}, van Eerten, H., MacFadyen, A., and Zhang, B.B. \\ 
	& ``Gamma Ray Bursts Are Observed Off-Axis.''  \emph{ApJ} 799 (2015) 3, [\href{https://arxiv.org/abs/1405.5516}{arXiv:astro-ph/1405.5516}] \\
\end{tabular}

\item \begin{tabular}{ll}
2015 & Zhang, B.B., van Eerten, H., Burrows, D., {\bf Ryan, G.}, {Evans}, P.A., Racusin, J., Troja, E., and MacFadyen, A. \\
 & ``Revisiting The GRB Jet-Break Problem With CHANDRA Deep Follow-up Data.''  \\
&  \emph{ApJ} 806 (2015) 15, [\href{https://arxiv.org/abs/1405.4867}{arXiv:astro-ph/1405.4867}] \\
\end{tabular}

\item \begin{tabular}{ll}
2011 & Penin, A., and {\bf Ryan, G.}  ``Two-Loop Electroweak Corrections To High Energy Large-Angle \\& Bhabha Scattering.'' \emph{JHEP} 1111 (2011) 081, [\href{https://arxiv.org/abs/1112.2171}{arXiv:hep-ph/1112.2171}] \\
\end{tabular}
\end{itemize}

\subsection*{White Papers and Proceedings}
\begin{itemize}

\item \begin{tabular}{ll}
2019 & {Kara}, E., {Margutti}, R., {Keivani}, A., {Fong}, W., {Cenko}, B., {Noble}, S., {Mushotzky}, R., {Burns}, E., \\
	& {\bf {Ryan}, G.}, {Ruan}, J.,  {Haggard}, D., {Burrows}, D., {Fox}, D., and {Caputo}, R. \\
	& ``{X-ray follow-up of extragalactic transients}'' Astro2020: Decadal Survey, Science White Paper no. 112 \\
	& \emph{BAAS} 51 (2019) 3 id.112 [\href{https://arxiv.org/abs/1903.05287}{arXiv:astro-ph/1903.05287}]
\end{tabular}
\item \begin{tabular}{ll}
2013 & {\bf Ryan, G.}, van Eerten, H., and MacFadyen, A. "Fitting Afterglows With Multi-Dimensional Simulations." \\& in Proceedings of the 7th Huntsville Gamma-Ray Burst Symposium, Nashville, Tennessee, USA, 2013, \\& edited by N. Gehrels, M. S. Briggs and V. Connaughton, eConf C1304143, 30, 2013 \\
\end{tabular}
\end{itemize}


\subsection*{Codes}
\begin{itemize}
\item \begin{tabular}{ll}
\texttt{afterglowpy} & Gamma ray burst afterglow models in Python, lead developer\\
& \\
\texttt{ScaleFit} & Gamma ray burst afterglow parameter estimation, lead developer\\
& \\
\texttt{disco} & Massively parallel moving mesh GRMHD, co-developer
\end{tabular}
\end{itemize}

% Proposals
\section*{Observing Proposals}
\begin{itemize}
\item \begin{tabular}{ll}
2020	& VLA Semester 20B ``The collimation and energetics of short gamma-ray bursts''\\
	&  PI Troja, 6 hrs
\end{tabular}
\item \begin{tabular}{ll}
2020	& VLA Semester 20B ``Electromagnetic counterparts to gravitational wave events''\\
	&  PI Troja, 29 hrs
\end{tabular}
\item \begin{tabular}{ll}
2020	& ATCA ``Late-time emergence of the radio kilonova of GW170817'' PI Piro
\end{tabular}
\item \begin{tabular}{ll}
2020	& GMRT Cycle 38 ``Electromagnetic counterparts of Gravitational Wave events: radio follow-up\\
	&  with uGMRT'' PI Troja, 4.5 hrs
\end{tabular}
\item \begin{tabular}{ll}
2019	& GMRT Cycle 37 `Electromagnetic counterparts of Gravitational Wave events: radio follow-up\\
	&  with uGMRT'' PI Troja, 20 hrs
\end{tabular}
\item \begin{tabular}{ll}
2019	& GMRT Cycle 36 `Electromagnetic counterparts of Gravitational Wave events: radio follow-up\\
	&  with uGMRT'' PI Troja, 9 hrs
\end{tabular}
\item \begin{tabular}{ll}
2018	& ATCA ``The relativistic outflow of GW170817 and the nature of the central remnant with ATCA''\\
	&  PI Piro
\end{tabular}
\item \begin{tabular}{ll}
2018	& ATCA ``Probing the relativistic outflow of GW170817 with ATCA'' PI Piro
\end{tabular}
\item \begin{tabular}{ll}
2018	& CXO Cycle 20 ``Electromagnetic Counterparts To Gravitational Wave Sources: A Synergistic\\
	&  Follow-Up With Chandra And The VLA'' PI Troja, 350 ks CXO \& 19.3 hrs VLA
\end{tabular}
\item \begin{tabular}{ll}
2018	& XMM-Newton AO 18  ``Observing The Gravitational Wave Sky With XMM-Newton''\\
	&  PI Piro, 330 ks XMM, 150ks NuSTAR, \& 3 hrs VLA
\end{tabular}
\end{itemize}

%Invited Talks  YET!
% \section*{Invited Talks}
\section*{Talks}
\begin{itemize}
\item \begin{tabular}{ll}
2020 & "Structured Jets At All Angles" Observational Astronomy Meeting  \\
	& CIERA Northwestern University, Evanston, Illinois, USA.  February 10
\end{tabular}
\item \begin{tabular}{ll}
2019 & "Structured Jets At All Angles" Astrophysics Seminar  \\
	& University of Bath, Bath, United Kingdom.  December 11
\end{tabular}
\item \begin{tabular}{ll}
2019 & "Structured Jets At All Angles" Director's Seminar  \\
	& Goddard Space Flight Center, Greenbelt, Maryland, USA.  October 11
\end{tabular}

\item \begin{tabular}{ll}
2018 & "The Afterglows Of Structured Jets" JSI Mini-Symposium  \\
	& University of Maryland, College Park, Maryland, USA.  November 2
\end{tabular}

\item \begin{tabular}{ll}
2017 & "Lighting Up Accretion Disks With Binary Black Holes" Astronomy Colloquium  \\
	& University of Maryland, College Park, Maryland, USA.  September 13
\end{tabular}

\item \begin{tabular}{ll}
2014 & "Gamma-Ray Bursts Are Observed Off-Axis." Institute for Theory and Computation Seminar  \\
	& Harvard-Smithsonian Center for Astrophysics, Cambridge, Massachusetts, USA.  September 23
\end{tabular}
\end{itemize}

%Conferences!
\section*{Conference and Workshop Participation}
\begin{itemize}

\item \begin{tabular}{ll}
2019 & Talk: ``Structured Jets At All Angles'' \\
	 &  30th Texas Symposium On Relativistic Astrophysics \\
	 &  Portsmouth, United Kingdom. December 15 - December 20
\end{tabular}

\item \begin{tabular}{ll}
2019 & Talk: ``Structured Jets At All Angles'' \\
	 &  Yamada Conference LXXI: Gamma-ray Bursts in the Gravitational Wave Era 2019 \\
	 &  Yokohama, Kanagawa, Japan. Oct 28 - Nov 1
\end{tabular}

\item \begin{tabular}{ll}
2019 & Poster: ``Structured Jets At All Angles'' \\
	 &  Astrophysics With Gravitational-Wave Populations \\
	 &  Aspen Center For Physics, Aspen, Colorado, USA. February 9-February 15
\end{tabular}

\item \begin{tabular}{ll}
2019 & Talk: ``Structured Jets and GW170817A'' \\
	 & Physics and Astrophysics at the eXtreme (PAX) V \\
	 & State College, Pennsylvania, USA. February 7
\end{tabular}

\item \begin{tabular}{ll}
2018 & Poster: ``The Afterglows Of Structured Jets'' \\
	 & JSI - Gravitational Wave Physics And Astronomy Workshop \\
	 & College Park, Maryland, USA. December 1-December 4
\end{tabular}

\item \begin{tabular}{ll}
2017 & Dissertation Talk: ``Black Hole Accretion Discs on a Moving Mesh'' \\
	 & American Astronomical Society Meeting 229 \\
	 & Grapevine, Texas, USA. January 3-January 7
\end{tabular}
\item \begin{tabular}{ll}
2016 & Time-Domain Astrophysics: Incorporating Observations, Theory, and Computation \\
	 & Radcliffe Institute for Advanced Study, Cambridge, USA. November 17-November 18
\end{tabular}
\item \begin{tabular}{ll}
2016 & Poster "Minidiscs in Circumbinary Black Hole Accretion" \\
	& 21st International Conference on General Relativity and Gravitation \\ 
	& Columbia University, New York City, USA. July 10-July 15
\end{tabular}

\item \begin{tabular}{ll}
2014 & International Summer School on Astro-Computing - Nuclear and Neutrino Astrophsyics \\ 
	& San Diego Supercomputing Center, University of California, San Diego, USA. July 21-August 1
\end{tabular}

\item \begin{tabular}{ll}
2013 & Poster "Fitting Afterglows With Multi-Dimensional Simulations." \\ 
	&7th Huntsville Gamma-Ray Burst Symposium, Nashville, Tennessee, USA.  April 14-18 \\
\end{tabular}

\item \begin{tabular}{ll}
2011 & Science Communication Workshop\\
	& Arthur L. Carter Journalism Institute, New York University, New York, USA.  Fall \\
\end{tabular}

\item \begin{tabular}{ll}
2010 & SLAC Summer Institute: Neutrinos - Nature's Mysterious Messengers, \\ 
	 & SLAC National Accelerator Laboratory, Menlo Park, California, USA.  August 2-13\\
\end{tabular}
\end{itemize}

%Departmental Talks
%\section*{Departmental Talks}

%Research Experience
%\section*{Research Experience}


%Awards
\section*{Fellowships and Awards}
\begin{itemize}

\item \begin{tabular}{ll}
2019 & Postdoctoral Prize For Excellence \\
& University of Maryland Department of Astronomy, May 2019 \\
\end{tabular}

\item \begin{tabular}{ll}
2017 & Joint Space-Science Institute Prize Fellowship \\
& University of Maryland and NASA Goddard Space Flight Center, Fall 2017 - Summer 2020 \\
\end{tabular}

\item \begin{tabular}{ll}
2015 & Dean's Dissertation Fellowship \\
& New York University, Fall 2015 - Summer 2016 \\
\end{tabular}

\item \begin{tabular}{ll}
2015 & Dean's Outstanding Graduate Student Teaching Award, New York University
\end{tabular}

\item \begin{tabular}{ll}
2014 & James Arthur Fellowship \\
& New York University, Fall 2014 - Summer 2015 \\
\end{tabular}

\item \begin{tabular}{ll}
2013 & James Arthur Fellowship \\
& New York University, Fall 2013 - Summer 2014 \\
\end{tabular}

\item \begin{tabular}{ll}
2011 & Henry M. MacCracken Fellowship \\
& New York University, Fall 2011 - Fall 2015 \\
\end{tabular}

\item \begin{tabular}{ll}
2011 & Graduate Student Teaching Award, University of Alberta \\
\end{tabular}

\item \begin{tabular}{ll}
2011 & Graduate Student Scholarship, Government of Alberta \\
\end{tabular}

%\item \begin{tabular}{ll}
%2008 & NSERC Undergraduate Summer Research Award
%\end{tabular}
%
%\item \begin{tabular}{ll}
%2008 & Jason Lang Award, University of Alberta
%\end{tabular}
%
%\item \begin{tabular}{ll}
%2007 & NSERC Undergraduate Summer Research Award
%\end{tabular}
%
%\item \begin{tabular}{ll}
%2007 & Academic Award, University of Alberta
%\end{tabular}
%
%\item \begin{tabular}{ll}
%2007 & Jason Lang Award, University of Alberta
%\end{tabular}
%
%\item \begin{tabular}{ll}
%2006 & NSERC Undergraduate Summer Research Award
%\end{tabular}
%
%\item \begin{tabular}{ll}
%2006 & Jason Lang Award, University of Alberta
%\end{tabular}
%
%\item \begin{tabular}{ll}
%2005 & Academic Excellence, University of Alberta
%\end{tabular}
%
%\item \begin{tabular}{ll}
%2005 & Entrance Scholarship, University of Alberta
%\end{tabular}

\end{itemize}

%Teaching Experience
\section*{Teaching Experience}
\subsection*{New York University}
\begin{itemize}
\item \begin{tabular}{ll}
2016 & Graduate Computational Physics, General Relativity \\
2014 & Graduate Computational Physics \\
2013 & Graduate Quantum Mechanics II \\
2012 & Graduate Quantum Mechanics I, Introductory Experimental Physics II
\end{tabular}
\end{itemize}
\subsection*{University of Alberta}
\begin{itemize}
\item \begin{tabular}{ll}
2011 & Physics 146 \\
2010 & Physics 144, Physics 126 \\
2009 & Physics 124 \\
\end{tabular}
\end{itemize}

%Teaching Experience
\section*{Outreach and Service}
\subsection*{University of Maryland}
\begin{itemize}
\item \begin{tabular}{ll}
2020			& GRAD-MAP Winter Workshop Mentor \\
2019			& GRAD-MAP Open House \& Site Visit Speaker \\
2019-		& Better Astronomy for the Next Generation (BANG) Seminar Organizing Committee \\
2017-2020 	& Department of Astronomy Equity, Diversity, and Inclusion Committee
\end{tabular}
\end{itemize}
\subsection*{NASA Goddard Space Flight Center}
\begin{itemize}
\item \begin{tabular}{ll}
2019- & Ask An Astrophysicist contributer
\end{tabular}
\end{itemize}


%Computers
%\section*{Computer Languages}
%\begin{itemize}
%\item C
%\item Python
%\item Java
%\item Mathematica
%\end{itemize}



\bigskip

% Footer
%\begin{center}
  %\begin{footnotesize}
   % Last updated: \today \\
    %\href{\footerlink}{\texttt{\footerlink}}
  %\end{footnotesize}
%\end{center}

\end{document}
